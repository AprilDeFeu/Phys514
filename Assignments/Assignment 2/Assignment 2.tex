\documentclass{article}


\usepackage{NotesStyle}

% Cover info

\title{Phys 514 \\
	\large Problem Set 2}

\author{April Sada Solomon - 260708051}
\date{Winter 2021}

\begin{document}
	\maketitle
	\newpage
	\tableofcontents
	\newpage
	
	\pagenumbering{arabic}
	\setcounter{page}{1}
	\cfoot{\thepage}
	
	\section{A bit of calculus}
		Consider $\R^3$ in the Cartesian coordinates $x^i = (x,y,z)$. Hmmmm... Wait a minute, something's not right. Ah, there we go. Consider $\R^3$ in the Cartesian coordinates $\ex^i = (\ex,\ey,\ez)$ with spherical coordinates $\ex^{i'} = (\er, \etheta, \ephi)$ as defined by
		\begin{align*}
			\ex &= \er \sin \etheta \cos \ephi \\
			\ey &= \er \sin \etheta \sin \ephi \\
			\ez &= \er \cos \etheta
		\end{align*}
		\subsection{Particle's path}
			Consider the world line $\ex^i (\elambda) = (\cos \elambda, \sin \elambda, \elambda)$. We want to express this world line in the spherical coordinates as defined above such that
			
			$$ \ex^i = \begin{pmatrix}
				\ex \\
				\ey \\
				\ez
			\end{pmatrix} = \begin{pmatrix}
			\cos \elambda \\
			\sin \elambda \\
			\elambda
			\end{pmatrix}$$
			If we wish to find a parametrization such that $(\ex, \ey, \ez) \to (\er, \etheta, \ephi)$ then it follows that
			\begin{align*}
				\er^2 &= \ex^2 + \ey^2 + \ez^2 \\
				\etheta &= \arctan \left( \frac\ey\ex \right) \\
				\ephi &= \arctan \left( \frac{\sqrt{\ex^2 + \ey^2}}{\ez} \right)
			\end{align*}
			So $\ex^2 + \ey^2 = \cos^2 \elambda + \sin^2 \elambda = 1$ implies that 
			\begin{align*}
				\er &= \pm \sqrt{\elambda^2 +1} \\
				\etheta &= \arctan \left( \frac{\sin \elambda}{\cos \elambda} \right) = \elambda \\
				\ephi &= \arctan \left( \frac{1}{\elambda} \right) = \cot^{-1} \elambda
			\end{align*}
			Hence, $\ex^i (\elambda) = (\sqrt{\elambda^2 + 1}, \elambda, \cot^{-1} \elambda)$.
			
		\subsection{Tangential Vector}
			Recall the tangential vector $T^i (\elambda) = \frac{d\ex^i}{d\elambda}$. Thus,
			$$ T^i (\elambda) = \frac{d}{d\elambda} \begin{pmatrix}
				\cos \elambda \\
				\sin \elambda \\
				\elambda
			\end{pmatrix} = \begin{pmatrix}
			-\sin \elambda \\
			\cos \elambda \\
			1
			\end{pmatrix}$$
			The Chain Rule becomes extremely useful when considering the change of coordinates $\ex^{i}$ such that
			$$ \bar{T}^{i} (\elambda) = \frac{d\bar{\ex}^{i}}{d \elambda} = \frac{d \bar{\ex}^{i}}{d \ex^j} \frac{d \ex^j}{d \et}$$
			where $\et$ represents time. Using the result from 1.a we get
			$$ \bar{T}^{i} (\elambda) = \frac{d}{d\elambda} \begin{pmatrix}
				\sqrt{\elambda^2 + 1} \\
				\elambda \\
				\cot^{-1} \elambda 
			\end{pmatrix} = 
			\begin{pmatrix}
				\frac{\elambda}{\sqrt{\elambda^2 + 1}} \\ 
				1 \\
				-\frac{1}{\elambda^2 + 1}
			\end{pmatrix}$$
		 	Spherical coordinates is easy in these cases. But what if we were the type of person that uses Arch Linux? Then we obviously would want to find the Cartesian coordinates:
		 	
		 	\begin{align*}
		 		\bullet \quad \bar{T}^{1} &= \frac{\partial \left( \sqrt{\ex^2 + \ey^2 + \ez^2} \right)}{\partial \ex} \left( -\sin \elambda \right) + \frac{\partial \left( \sqrt{\ex^2 + \ey^2 + \ez^2} \right)}{\partial \ey} \left( \cos \elambda \right) \\
		 		&\longspace + \frac{\partial \left( \sqrt{\ex^2 + \ey^2 + \ez^2} \right)}{\partial \ez} \left( 1 \right) \\
		 		&= \frac{\ex}{\er} \left( -\sin \elambda \right) + \frac{\ey}{\er} \left(\cos \elambda \right) + \frac{\ez}{\er} (1) \\
		 		&= - \frac{\cos \elambda \sin \elambda}{\er} + \frac{\cos\elambda \sin \elambda}{\er} + \frac{\ez}{\er} \\
		 		&= \boxed{\frac{\elambda}{\sqrt{\elambda^2 + 1}} }\\
		 		\bullet \quad \bar{T}^2 &= \frac{\partial \arctan \left( \frac\ey\ez \right)}{\partial \ex}(-\sin \elambda) + \frac{\partial \arctan \left( \frac\ey\ez \right)}{\partial \ey} \cos \elambda + \frac{\partial \arctan \left( \frac\ey\ez \right)}{\partial \ez} (1) \\
		 		&= \frac{\ey}{\ex^2 + \ey^2} \sin \elambda + \frac{\ex}{\ex^2 + \ey^2} \cos \elambda \\
		 		&= \boxed{ 1 }
		 	\end{align*}
	 		\begin{align*}
	 			\bullet \quad \bar{T}^3 &= \frac{\partial \arctan \left( \frac{\sqrt{\ex^2 +\ey^2}}{\ez} \right)}{\partial \ex} (-\sin \elambda) + \frac{\partial \arctan \left( \frac{\sqrt{\ex^2 +\ey^2}}{\ez} \right)}{\partial \ey} (\cos \elambda) \\
	 			&\longspace + \frac{\partial \arctan \left( \frac{\sqrt{\ex^2 +\ey^2}}{\ez} \right)}{\partial \ez} (1) \\
	 			&= -\frac{\ex \ez \sin \elambda}{\sqrt{\ex^2 + \ey^2} \er^2} +\frac{\ey \ez \cos \elambda}{\sqrt{\ex^2 + \ey^2} \er^2} - \frac{\sqrt{\ex^2 + \ey^2}}{\er^2} \\
	 			&= \frac{\ez}{\sqrt{\ex^2 + \ey^2} \er} \left( - \cos \elambda \sin \elambda + \sin \elambda \cos \elambda \right) - \frac{\sqrt{\ex^2 + \ey^2}}{\er} \\
	 			&= \boxed{-\frac{1}{\elambda^2 + 1}} 
	 		\end{align*}
 			Therefore:
 			\begin{align*}
 				T^i (\elambda) &= \begin{pmatrix}
 					\cos \elambda \\
 					\sin \elambda \\
 					\elambda
 				\end{pmatrix} &
 				\bar{T}^i (\elambda) &= \begin{pmatrix}
 					\frac{\elambda}{\sqrt{\elambda^2 + 1}} \\ 
 					1 \\
 					-\frac{1}{\elambda^2 + 1}
 					\end{pmatrix}
 			\end{align*}
 		\pagebreak
 		\section{Vector fields}
 			\subsection{Infinitesimal vector translations}
 			Alright, no more emojis for this assignment $\ex$. Let $\dot{x} = \dot{x}^\mu d\mu = \frac{\partial \alpha^\mu}{\partial \lambda}\partial_\mu$ where $\partial_\mu = (\partial_1, \partial_2, \partial_3)$. Thus, it follows that
 			$$ x^\mu \to \alpha^\mu (x^\mu, \lambda)$$ 
 			Now consider the transformation
 			$$ T = I + \lambda \dot{x}$$
 			Then we have
 			\begin{align*}
 				\alpha &= T x \\
 				&= (I + \lambda \dot{x}) x \\
 				&= \left( I + \lambda \frac{\partial \alpha^\mu}{\partial \lambda \frac{\partial}{\partial x^\mu}} \right)x \\
 				&= x + \lambda \frac{\partial \alpha^\mu}{\partial \lambda \frac{\partial x}{\partial x^\mu}} \\
 				&= x + \lambda \frac{\partial \alpha^\mu}{\partial \lambda} \\
 				&= x + \lambda \dot{x}^\mu
 			\end{align*} 
 			As expected, we have $\alpha^\mu = x^\mu + \lambda \dot{x}^\mu$. We can now change coordinates such that
 			\begin{align*}
 				\alpha^i &= x^i + \lambda \frac{\partial \alpha^i}{\partial \lambda} \frac{\partial x^i}{\partial x^\mu} \\
 				\intertext{where we have $\partial_i = \frac{\partial}{\partial x^i} $ for $i=1,2,3$. Thus:}
 				\alpha^i &= x^i + \lambda \frac{\partial \alpha^i}{\partial \lambda} (x^1, x^2, x^3)
 				\intertext{where $i \neq 1 \lor 2 \lor 3 \implies x^1 \lor x^2 \lor x^3 = 0$, and $i = 1 \lor 2 \lor 3 \implies x^1 \lor x^2 \lor x^3 = x^i$. So for $i=1$ we have $\frac{\partial x^1}{\partial x^1} = (x^1, 0,0)$, for $i=2$ we have $\frac{\partial x^i}{\partial x^i} = (x^1, 0,0)$, and for $i=3$ we have $(0,0,x^3)$. Hence, considering $x^1$:}
 				\alpha^1 &= x^1 + \lambda \frac{\partial \alpha^i}{\partial \lambda} (x^1, 0, 0) \\
 				&=    x^1 \left( 1 + \lambda \frac{\partial \alpha^1}{\partial \lambda} \right)
 			\end{align*}
 			As $\lambda \to 0$, it is clear that $\alpha^1$ represents an infinitesimal displacement of $x^1$ on the $x^1$ axis. It follows similarly for $i=2$ and $i=3$.
 			
 			\subsection{Infinitesimal vector rotations}
 			Given
 			$$
 			\alpha^\mu = 
 			\begin{pmatrix}
 				v\\
 				w\\
 				z
 			\end{pmatrix} =
 			\begin{pmatrix}
 				0 & x^2 \partial_3  & - x^3 \partial_2 \\
 				x^3 \partial_1 & 0 & - x^1 \partial_3 \\
 				x^1 \partial_2  &- x^2 \partial_1 & 0 \\
 			\end{pmatrix}
			$$
			we have the following rotation matrices 
			\begin{align*}
				R_{x^1}(\theta) &= 
					\begin{pmatrix}
						1		&		0		&		0		\\
						0		&	\cos \theta		&	-\sin \theta		\\
						0		&	\sin \theta		&	\cos \theta
					\end{pmatrix}
					\approx
					\begin{pmatrix}
						1		&		0		&		0		\\
						0		&	1		&	 -\theta		\\
						0		&	\theta		&	 1
					\end{pmatrix} \\
				R_{x^2} (\theta) &= 
					\begin{pmatrix}
						\cos \theta	&		0		&	\sin \theta		\\
						0		&		1		&	0			\\
						-\sin \theta	&		0		&	\cos \theta
					\end{pmatrix}
					\approx
					\begin{pmatrix}
						1		&		0		&		\theta		\\
						0		&	1		&	 0		\\
						-\theta		&	0		&	 1
					\end{pmatrix} \\
				R_{x^1}(\theta) &= 
					\begin{pmatrix}
						\cos \theta	&	-\sin \theta		&		0		\\
						\sin \theta		&	\cos \theta		&	0		\\
						0		&	0		&	1
					\end{pmatrix}
					\approx
					\begin{pmatrix}
						1		&		-\theta		&		0		\\
						\theta		&	1		&	 0		\\
						0		&	0		&	 1
					\end{pmatrix} 
			\end{align*}
			for $\theta \to 0$ angle of rotation. The image of $\Omega = \begin{psmallmatrix} v\\w\\z \end{psmallmatrix}$ is thus given by
			$$ \text{Image} (\Omega) = \begin{pmatrix}
				0	&	-r^3	&	r^2	\\
				r^3	&	0		&	-r^1 \\
				 -r^2	&	r^1	&	0
			\end{pmatrix}$$
 			Therefore,
 			\begin{align*}
 				x^{\mu'} &= \text{Image}(\Omega) x^\mu \\
 				&= \begin{pmatrix}
 						0	&	-r^3	&	r^2	\\
 						r^3	&	0		&	-r^1 \\
 						-r^2	&	r^1	&	0
 					\end{pmatrix}
 					\begin{pmatrix}
 						x^1 \\
 						x^2 \\
 						x^3
 					\end{pmatrix}\\
 				&= \begin{pmatrix}
 					0	&	-r^3 x^2 & r^2 x^3 \\
 					r^3 x^1 & 0 & -r^1 x^3 \\
 					-r^2 x^1 & r^1 x^2 & 0
 				\end{pmatrix}
 			\end{align*}
 			which can be rearranged to $\alpha^\mu$ by replacing in $-r^i = \partial_i$, as expected.
 			
 		\subsection{Radial coordinate $r$}
 		\subsubsection{Dot products}
 			It is clear that
 			$$ r^2 = (x^1)^2 + (x^2)^2 + (x^3)^2$$
 			So the dot products are given by:
 			\begin{itemize}
 				\item $ $\vspace{-1cm}
 					\begin{align*}
 						vr &= x^2 \frac{\partial}{\partial x^3} \sqrt{(x^1)^2 + (x^2)^2 + (x^3)^2} - x^3 \frac{\partial}{\partial x^2} \sqrt{(x^1)^2 + (x^2)^2 + (x^3)^2} \\
 						&= \frac{x^2 x^3}{r} - \frac{x^3 x^2}{r} = 0
 					\end{align*}
 				\item $ $\vspace{-1cm}
 					\begin{align*}
 						wr &= x^3 \frac{\partial}{\partial x^1} \sqrt{(x^1)^2 + (x^2)^2 + (x^3)^2} - x^1 \frac{\partial}{\partial x^3} \sqrt{(x^1)^2 + (x^2)^2 + (x^3)^2} \\
 						&= \frac{x^3 x^1}{r} - \frac{x^1 x^3}{r} = 0
 					\end{align*}
 				\item $ $\vspace{-1cm}
	 				\begin{align*}
	 					zr &= x^1 \frac{\partial}{\partial x^2} \sqrt{(x^1)^2 + (x^2)^2 + (x^3)^2} - x^2 \frac{\partial}{\partial x^1} \sqrt{(x^1)^2 + (x^2)^2 + (x^3)^2} \\
	 					&= \frac{x^1 x^2}{r} - \frac{x^2 x^1}{r} = 0
	 				\end{align*}
 			\end{itemize}
 			As we saw in the rest of Problem 2, $x^\mu \times \partial_\mu$ represents a rotation about an axis in the coordinate system. As rotations change the direction of a vector but not its magnitude, we conclude that:
 			$$ (x^\mu \times \partial_\mu)r = \vec{0}$$
\end{document}