\documentclass{article}


\usepackage{NotesStyle}

% Cover info

\title{Phys 514 \\
	\large Problem Set 3}

\author{April Sada Solomon - 260708051}
\date{Winter 2021}

\begin{document}
	\maketitle
	\thispagestyle{empty}
	\pagebreak
	
	\pagenumbering{roman}
	\cfoot{\thepage}
	
	\tableofcontents
	\newpage
	
	% Start page count after the TOC
	
	\pagenumbering{arabic}
	\setcounter{page}{1}
	\cfoot{\thepage}
	% Notes body
	\section{Tensors on Manifolds}
		Consider a $D$-dimensional manifold. In any coordinate system $x^\mu$, the Kronecker delta symbol $\delta_\nu^\mu$ is defined to be
		$$ \delta_\nu^\mu = \begin{cases}
			1 & \text{if} \quad \mu = \nu  \\
			0 & \text{if} \quad \mu \neq \nu
		\end{cases}$$
		\subsection{Tensor weight}
			We need to show that $\delta_\nu^\mu$ is a tensor of weight $(1,1)$. We apply the tensor transformation laws such that $\delta_{\nu'}^{\mu '}$ is a Kronecker delta over a new set of coordinates $x^{\mu'}$ given by
			\begin{align*}
				\delta_{\nu'}^{\mu '} &= \delta_{\nu}^\mu \frac{\partial x^{\mu'}}{\partial x^\mu} \frac{\partial x^\nu}{\partial x^{\nu '}} \\
			\intertext{We notice that we only consider the terms where $\mu = \nu$, as they will be the only non-zero terms. We can thus define $\mu=\nu=\alpha$ such that:}
				\delta_{\nu'}^{\mu '} &= \sum_\alpha \frac{\partial x^{\mu'}}{\partial x^\alpha} \frac{\partial x^\alpha}{\partial x^{\nu '}} \\
				&= \sum_\alpha \frac{\partial x^\alpha}{\partial x^{\nu '}}  \frac{\partial x^{\mu '}}{\partial x^\alpha} \\
				&= \frac{\partial x^{\mu '}}{\partial x^{\nu'}} \\
				&= 
					\begin{cases}
						1 & \text{if } \mu' = \nu' \\
						0 & \text{if } \mu' \neq \nu' 
					\end{cases} \\
				&= \delta_{\nu}^\mu
			\end{align*}
			Now, considering $\hat{e_\mu}$ as the basis vectors for coordinates $x^\mu$, and $\hat{e_{\nu'}}$ as the basis vectors for coordinates $x^{\nu'}$, it follows that for any transformation of coordinates $\Lambda^{\nu'}_{\mu}$ we have
			$$ \hat{e}_\mu = \Lambda_\mu^{\nu'} \hat{e}_{\nu'}$$
			And to isolate the basis $\hat{e}_{\nu'}$, we must apply the inverse of $\Lambda_{\mu}^{\nu'}$, namely $\Lambda_{\sigma'}^\rho$ such that
			$$ \Lambda_{\nu'}^{\mu}\Lambda_{\rho}^{\nu'} = \delta_\rho^\mu, \quad \quad \Lambda_{\lambda}^{\sigma'} \Lambda_{\tau'}^{\lambda} = \delta_{\tau'}^{\sigma'}$$
			Which clearly implies that
			$$ \hat{e}_{\nu'} = \Lambda^\mu_{\nu'} \hat{e}_\mu$$
			So $\delta_{\nu}^\mu$ is a tensor of weight $(1,1)$, as it has a covariant index and a contravariant index.
		\subsection{Vectors and tensors}
			We need to show that $\delta_\nu^\mu v^\nu = v^\mu$ and $\delta_{\nu}^{\mu} w_\mu = w_\nu$ where $v^\nu$ and $w_\mu$ are a vector and one form respectively. As seen in the second part of my answer to $1.1$, $\delta_{\nu}^\mu$ represents a transformation multiplied by its own inverse. As such, we can clearly see that
			$$ \delta_\nu^\mu v^\nu = \Lambda_{\nu'}^\mu \Lambda_{\nu}^{\nu'} v^\nu = \Lambda_{\nu'}^{\mu} ( \Lambda_{\nu}^{\nu'}  v^\nu) = \Lambda_{\nu'}^{\mu} v^{\nu'} = v^\mu$$
			$$ \delta_\nu^\mu w_\mu = \Lambda_{\nu'}^\mu \Lambda_{\nu}^{\nu'} w_\mu = \Lambda_{\nu}^{\nu'}(\Lambda_{\nu'}^{\mu}w_\mu)  = \Lambda_{\nu}^{\nu'} w_{\nu'} = w_\nu  $$
			As required.
		\subsection{Computation}
			First we have $\delta_\mu^\mu$, which seems very sneaky. A quick guess would assume that 
			$$ \delta_\mu^\mu = \begin{cases}
				1 & \text{if} \quad \mu = \mu \\
				0 & \text{if} \quad \mu \neq \mu
			\end{cases}$$
			But aha, this is not what we should see. In this case, we should add the number of dimensions whenever the indices of the Kronecker delta are equal (which they always are), so written in transformations:
			$$ \delta_\mu^\mu = \Lambda_{\nu'}^\mu \Lambda_\mu^{\nu'}= \sum_i^D \frac{\partial x^i}{\partial x^i} =D$$
			
			Next we have $\delta_\nu^\mu \delta_\rho^\mu$. We again write this down in transformations:
			\begin{align*}
				\delta_\nu^\mu \delta_\rho^\mu = \Lambda_{\nu'}^\mu \Lambda_\nu^{\nu'} \Lambda_{\sigma'}^\mu \Lambda_\rho^{\sigma'}
			\end{align*}
			Which means that
			$$ \delta_\nu^\mu \delta_\rho^\mu = \begin{cases}
				D & \text{if} \quad \mu = \nu = \rho \\
				0 & \quad \text{otherwise}
			\end{cases}$$
			So we will only add the elements whenever the indices match, and they will add up to the number of dimensions of the space.
		\pagebreak
	\section{General Questions}
		\subsection{Dummy (\textbf{t h i c c}) indices}
			We must show by manipulating dummy indices that
			$$ Z_{\mu\nu}v^\mu v^\nu = \frac12 \left(Z_{\mu\nu}+Z_{\nu\mu} \right)v^\mu v^\nu$$
			where we assume $Z_{\mu\nu}$, $Z_{\nu\mu}$ are both rank (0,2) tensors that are also symmetric matrices. Hence, we know that $Z_{\mu\nu} = Z_{\nu\mu}^T$ (see equation 1.69 in the book for an example, where the tensor is antisymmetric instead of symmetric), so clearly:
			\begin{align*}
				\frac{1}{2} (Z_{\mu\nu} + Z_{\nu\mu})  &= \frac12 \left(
				\begin{pmatrix}
					z_{1, 1} & \cdots &  z_{1, \nu} \\
					\vdots 	&	\ddots &	\vdots \\
					z_{\mu, 1} & \cdots & z_{\mu, \nu}
				\end{pmatrix}
				+ \begin{pmatrix}
					z_{1, 1} & \cdots &  z_{1, \mu} \\
					\vdots 	&	\ddots &	\vdots \\
					z_{\nu, 1} & \cdots & z_{\nu, \mu}
				\end{pmatrix} \right) \\
				&= \frac12 
				\begin{pmatrix}
					2\cdot z_{1, 1} & \cdots &  z_{1, \nu}+z_{1, \mu} \\
					\vdots 	&	\ddots &	\vdots \\
					z_{\mu, 1}+z_{\nu,1} & \cdots & \underbrace{2 z_{\mu, \nu}}_{=2z_{\nu, \mu}}
				\end{pmatrix}\\
				 &= \begin{pmatrix}
				 	z_{1, 1} & \cdots &  \frac{z_{1, \nu}+z_{1, \mu}}{2} \\
				 	\vdots 	&	\ddots &	\vdots \\
				 	\frac{z_{\mu, 1}+z_{\nu,1}}{2} & \cdots &  z_{\mu, \nu}
				 \end{pmatrix} = \begin{pmatrix}
				 z_{1, 1} & \cdots &  \frac{z_{1, \mu}+z_{1, \nu}}{2} \\
				 \vdots 	&	\ddots &	\vdots \\
				 \frac{z_{\nu, 1}+z_{\mu,1}}{2} & \cdots &  z_{\mu, \nu}
				 \end{pmatrix} \\
		 		\intertext{Recall that both tensors are symmetric}
			 	&= \begin{pmatrix}
			 		z_{1, 1} & \cdots &  z_{1, \nu} \\
			 		\vdots 	&	\ddots &	\vdots \\
			 		z_{\mu, 1} & \cdots & z_{\mu, \nu}
			 	\end{pmatrix}\\
		 		&= Z_{\mu\nu}
			\end{align*}
			Therefore,
			$$ Z_{\mu\nu}v^\mu v^\nu = \frac12 \left(Z_{\mu\nu}+Z_{\nu\mu} \right)v^\mu v^\nu$$
			and the dummy index is thicc indeed.
			
		\pagebreak	
		\subsection{Independent components}
			For any (0,2) tensor, we have 2 covariant indices and no contravariant indices. Hence, assuming the tensor is non-symmetric to distinguish this case from symmetric and antisymmetric, we can simply represent the number of independent components in any rank $(0,2)$ tensor of dimension $D$ as the total number of components in the tensor, so this is given by $D^2$
			
			Now if we consider symmetric tensors, then we know that $T_{ij} = T_{ji}$, so we may only consider $\sum_{k=1}^{D} k$ independent elements. In 3D, this yields 1+2+3 = 6, in 4D we have 1+2+3+4 = 10, etc.
			
			For an antisymmetric tensor where $T_{ij} = -T_{ji}$ for $i\neq j$ and $T_{ij} = 0$ for $i=j$, we consider the same amount of independent components as the symmetric case but without the diagonal, so we have $\sum_{k=1}^{D-1} k$ independent components. In 3D we have 1+2 =3 independent component, in 4D we have 1+2+3 = 6, and so on.
		\subsection{Sum of tensors}
			As we saw in part 2.2, an antisymmetric tensor will contain independent elements $T_{ij} = -T_{ji}$ except $T_{ij} = 0$ for $i=j$, while a symmetric tensor need only have the property that $T_{ij} = T_{ji}$ and $T_{ij} \neq 0$ for $i=j$. Thus, we can build a (0,2) tensor with $D^2$ independent components by adding a symmetric matrix and an antisymmetric matrix together. This will yield a new tensor where we will have the component property $T_{ij} \neq 0$ for $i=j$ and also $T_{ij} \neq T_{ji} \neq -T_{ji}$.
		\subsection{Independent components pt. 2}
			This case, we are talking about an outer product of two $(0,1)$ tensors, which yields a $(0,2)$ tensor as desired. Thus, we can evidently see that we would have $2D - 1$ where we remove one degree of freedom as the determinant would be zero. So for any $D \geq 2$ dimensions, we will have a (0,2) tensor resulting from 2 (0,1) one forms.
		\subsection{Independent components (bonus)}
			For $(0,p)$ tensors, we would have $D^p$ independent components in the general case, $\sum_{k=1}^D k^{p-1}$ for the symmetric case, and $\sum_{k=1}^{D-1} k^{p-1}$ for the antisymmetric case. We extend the logic from 2.2 such that in a symmetric tensor we have $T_{ijk} = T_{jik} = T_{jki} = T_{kji} = T_{ijk}$ and also equals all permutations of $ijk$ in $T_{ijk}$. In an antisymmetric tensor, we'd have $T_{ijk} = - T_{jik} = T_{jki} = -T_{kji} = T_{kij} = -T_{ikj}$, so clearly, when adding a symmetric and antisymmetric tensor we would yield a (0,3) nonsymmetric tensor. And this can be extended to any $(0,p)$ tensor as well, as every general nonsymmetric tensor can be decomposed into an antisymmetric and a symmetric part.
	\section{Higher-dimensional Space}
		\subsection{Transformations}
			When we consider transformations on one forms $w$, it follows that
			$$ w_\mu \to w_{\mu'} = \frac{\partial x^\mu}{\partial x^{\mu'}} w_\mu$$
			Where we have that $\frac{\partial x^\mu}{\partial x^{\mu'}}$ is the inverse of the Jacobian, so therefore we have that $\exists \frac{\partial x^{\mu'}}{\partial x^\mu}$ such that
			$$ \frac{\partial x^{\mu}}{\partial x^{\mu'}}\frac{\partial x^{\mu'}}{\partial x^{\mu}} = \frac{\partial x^{\mu}}{\partial x^{\mu'}}\frac{\partial x^{\mu'}}{\partial x^{\mu}} = I$$
			As we had in part 1.1. If we would like $v_a$ to transform like a one form, then it is necessary that we have
			\begin{equation}
				\label{eq:1}
				v_{a'} = \frac{\partial y^a}{\partial y^{a'}} v_a
			\end{equation}
			Now, let us assume that $v_i$ is a one form and hence transforms like a one form. We have that
			$$ v_{i'} = \frac{\partial x^i}{\partial x^{i'}} v_i$$
			$$ \therefore \frac{\partial x^{i'}}{\partial x^{i}}v_{i'} = \frac{\partial x^{i'}}{\partial x^{i}} \frac{\partial x^i}{\partial x^{i'}} v_i = v_i$$
			We can now use the above conclusion in equation (\ref{eq:1}), hence:
			\begin{align*}
				v_{a'} &= \frac{\partial y^a}{\partial y^{a'}} v_a
				= \frac{\partial y^a}{\partial y^{a'}} \left( \frac{\partial x^i}{\partial y^a} v_i \right) 
				\intertext{By the chain rule on Jacobians, we have}
				&= \frac{\partial x^i}{\partial y^{a'}} v_i 
				= \frac{\partial x^i}{\partial y^{a'}} \left( \frac{\partial x^{i'}}{\partial x^i} v_{i'} \right)
				\intertext{again, by chain rule}
				&= \frac{\partial x^{i'}}{\partial y^{a'}} v_{i'}
			\end{align*}
			So we can conclude that for $v_i = \frac{\partial x^i}{\partial y^a} v^a$ under one form transformation, $v_a$ clearly transforms like a one form.
		\pagebreak
		\subsection{Transformations 2}
			We follow through as in part 3.1 with the same transformation, but on vectors instead of one forms. We have
			\begin{align*}
				v^{i'} &= \frac{\partial x^{i'}}{\partial x^{i}} v^i \\
				v^{a'} &= \frac{\partial y^{a'}}{\partial y^{a}} v^a
			\end{align*}
			So therefore,
			\begin{align*}
				v^{i'} &= \frac{\partial x^{i'}}{\partial x^{i}} v^i
				= \frac{\partial x^{i'}}{\partial x^{i}} \left( \frac{\partial x^i}{\partial y^a} v^a \right) 
				\intertext{By the chain rule on Jacobians, we have}
				&= \frac{\partial x^{i'}}{\partial y^{a}} v^a 
				= \frac{\partial x^{i'}}{\partial y^{a}} \left( \frac{\partial y^{a'}}{\partial y^{a'}} v^{a} \right)
				\intertext{again, by chain rule}
				&= \frac{\partial x^{i'}}{\partial y^{a'}} v^{a'}
			\end{align*} 
			So  $v^i$ is a vector that transforms like a vector on $\M$.
		\pagebreak
		\subsection{Metric on Cartesian Coordinates}
			We have
			\begin{align*}
				dx^1 &= d(\sin \theta \cos \phi) = (\cos\theta \cos \phi) d\theta - (\sin \theta \sin \phi)d \phi \\
				dx^2 &= d(\sin \theta \sin \phi) = (\cos \theta \sin \phi) d\theta + (\sin \theta \cos \phi) d \phi \\
				dx^3 &= d(\cos\theta) = -\sin \theta d\theta
			\end{align*}
			Therefore
			$$ ds^2 = (dx^1)^2+(dx^2)^2+(dx^3)^2$$
			Where
			\begin{align*}
				(dx^1)^2 &= (\cos\theta \cos \phi d \theta - \sin \theta \sin \phi d \phi)^2 \\
				&= \cos^2 \theta \cos^2 \phi d\theta^2 - 2 \cos \theta \sin \theta \cos \phi \sin \phi d \phi d \theta + \sin^2\theta \sin^2\phi d\phi^2 \\
				(dx^2)^2 &= (\cos\theta \sin \phi d \theta - \sin \theta \cos \phi d \phi)^2 \\
				&= \cos^2 \theta \sin^2 \phi d\theta^2 + 2 \cos \theta \sin \theta \cos \phi \sin \phi d \phi d \theta + \sin^2\theta \cos^2\phi d\phi^2 \\
				(dx^3)^2 &= (-\sin\theta d \theta)^2 = \sin^2 \theta d \theta^2
			\end{align*}
			Thus
			\begin{align*}
				ds^2 &= (dx^1)^2+(dx^2)^2+(dx^3)^2 \\
				&= d\theta^2 \left( \cos^2 \theta \cos^2 \phi + \cos^2 \theta \sin^2 \phi + \sin^2 \theta \right) + d\phi^2 \left( \sin^2 \theta \sin^2 \phi + \sin^2 \theta \cos^2 \phi \right) \\
				&= d\theta^2 \left( \cos^2 \theta (\cos^2 \phi + \sin^2 \phi) + \sin^2 \theta \right) + d\phi^2 \left( \sin^2 \theta (\sin^2 \phi + \cos^2 \phi) \right) \\
				&= d\theta^2 \left( \cos^2\theta + \sin^2 \theta \right) + \sin^2 \theta d\phi^2 \\
				&= d\theta^2 + \sin^2\theta d\phi^2
			\end{align*}
			Where we have $f(\theta) = \sin^2 \theta$.
			\pagebreak
		\subsection{Equivalent metrics}
			Consider the trigonometric rotations by $\nicefrac\pi2$: $\cos\theta = \sin (\theta + \nicefrac\pi2)$ and $-\sin\theta = \cos (\theta + \nicefrac\pi2)$. From the previous question, we have
			\begin{align*}
				dx^1 &= d(\cos \theta \cos \phi) = -(\sin\theta \cos \phi) d\theta - (\cos \theta \sin \phi)d \phi \\
				dx^2 &= d(\cos \theta \sin \phi) = -(\sin \theta \sin \phi) d\theta + (\sin \theta \cos \phi) d \phi \\
				dx^3 &= d(-\sin\theta) = -\cos \theta d\theta
			\end{align*}
			where $ ds^2 = (dx^1)^2+(dx^2)^2+(dx^3)^2$. So again,
			\begin{align*}
				(dx^1)^2 &= (-\sin\theta \cos \phi d \theta - \cos \theta \sin \phi d \phi)^2 \\
				&= \sin^2 \theta \cos^2 \phi d\theta^2 + 2 \cos \theta \sin \theta \cos \phi \sin \phi d \phi d \theta + \cos^2\theta \sin^2\phi d\phi^2 \\
				(dx^2)^2 &= (-\sin\theta \sin \phi d \theta - \cos \theta \cos \phi d \phi)^2 \\
				&= \sin^2 \theta \sin^2 \phi d\theta^2 - 2 \cos \theta \sin \theta \cos \phi \sin \phi d \phi d \theta + \cos^2\theta \cos^2\phi d\phi^2 \\
				(dx^3)^2 &= (-\cos\theta d \theta)^2 = \cos^2 \theta d \theta^2
			\end{align*}
			Hence,
			\begin{align*}
				ds^2 &= (dx^1)^2+(dx^2)^2+(dx^3)^2 \\
				&= d\theta^2 \left( \sin^2 \theta \cos^2 \phi + \sin^2 \theta \sin^2 \phi + \cos^2 \theta \right) + d\phi^2 \left( \cos^2 \theta \sin^2 \phi + \cos^2 \theta \cos^2 \phi \right) \\
				&= d\theta^2 \left( \sin^2 \theta (\cos^2 \phi + \sin^2 \phi) + \cos^2 \theta \right) + d\phi^2 \left( \cos^2 \theta (\sin^2 \phi + \cos^2 \phi) \right) \\
				&= d\theta^2 \left( \sin^2\theta + \cos^2 \theta \right) + \cos^2 \theta d\phi^2 \\
				&= d\theta^2 + \cos^2\theta d\phi^2
			\end{align*}
			Which is again only a rotation by $\nicefrac\pi2$
	\pagebreak
	\section{Minkowski metric problem}
		We again do the same as for the previous questions in part 3 such that
		\begin{align*}
			dx^0 &= \cosh(t) dt \\
			dx^1 &= (\sinh(t)\cos \phi) dt - (\cosh (t)\sin \phi)d \phi \\
			dx^2 &= -(\sinh (t) \sin \phi) dt + (\cosh (t) \cos \phi) d \phi \\
		\end{align*}
		So $$ ds^2 = - (dx^0)^2 + (dx^1)^2+(dx^2)^2 = \eta_{\mu\nu} dx^\mu dx^\nu$$
		such that
		\begin{align*}
			ds^2 &= (dx^0)^2+(dx^1)^2+(dx^2)^2 \\
			&= -\cosh^2 (t) dt^2 + \sinh^2 (t) \cos^2 \phi dt^2 - \sinh (t) \cosh (t) \sin \phi \cos \phi dt d\phi + \cosh^2 (t) \sin^2 \phi d\phi^2  \\
			&\quad\quad\quad\quad + \sinh^2 (t) \sin^2 \phi dt^2 + \sinh (t) \cosh (t) \sin \phi \cos \phi dt d\phi + \cosh^2 (t) \cos^2 \phi d\phi^2\\
			&= dt^2 \left( \sinh^2 (t) (\cos^2 \phi + \sin^2 \phi) - \cosh^2 (t) \right) + d\phi^2 \left( \cosh^2 (t) (\sin^2 \phi + \cos^2 \phi) \right) \\
			&= dt^2 \left( \sinh^2(t) - \cosh^2 (t) \right) + \cos^2 (t) \phi^2 \\
			&= -dt^2 + \cosh^2(t) d\phi^2
		\end{align*}
		So we obtain the metric on $dS_2$ as $-dt^2 + \cosh^2(t) d\phi^2$ and
		$$ g_{\mu\nu} = \text{diag}(-1,\cosh^2(t))$$
		Furthermore, we know that $\eta_{\mu\nu} =$diag$(-1,1,1)$ implies that $\eta_{\mu\nu} x^\mu x^\nu = 1$, so
		$$ \eta_{\mu\nu} x^\mu x^0 = -(x^0)^2 + (x^1)^2 + (x^2)^2$$
		Hence
		\begin{align*}
			\eta_{\mu\nu} x^\mu x^0 &= - \sinh^2 (t) + \cosh^2(t) \cos^2 \phi + \cosh^2(t) \sin^2 \phi \\
			&= \cosh^2(t) - \sinh^2(t) \\
			&= 1
		\end{align*}
		Which implies a valid parametrization such that
		$$ \eta_{\mu\nu} x^\mu x^0 = 1 $$
		
\end{document}